\documentclass{article}
\usepackage[utf8]{inputenc}

\title{Problem 1: A Brief Description of PI}
\author{Zan Wang}
\date{July 5th, 2019}

\begin{document}

\maketitle

\section{Basic Definition}
PI is represented by the Greek letter $\Pi$ and is a constant, approximately 3.14159265 representing the ratio of the circumference to the diameter. It is an irrational number, that is, infinite loops.
\section{Practice and History}
Calculating the value of the PI has its special meaning. However, PI with accuracy of dozens digits used in the field of modern technology are sufficient. If the size of the observable universe is calculated with a 39-degree precision PI value, the error is less than the volume of one atom. Previous people calculated PI to find out whether the PI is a circular fraction. Since 1761, Lambert proved that PI is irrational. Then in 1882, Lindeman proved that PI is the transcendental number, and the mystery of the PI is unveiled.
\section{Others}
Besides, in 2009, the US House of Representatives formally adopted a non-binding resolution, setting March 14th of each year as the “PI-day”.


\end{document}
