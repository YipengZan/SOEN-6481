\chapter{Interview: Problem 2}
\vspace{8pt}

\section{Session of Question and Answer}
 \begin{itemize}

\item[1)] Do you need to deal with numbers frequently?
\item[-]Answer 1: Yes
\item[-] Answer 2: Sure 

\item[2)]How often do you use a calculator?
\item[-]Answer 1: Very often, but just some very easy calculation
\item[-]Answer 2: Sure...unless I must use my head

\item[3)]When and where do you use it?
\item[-]Answer 1: Actually, mostly I could satisfy my needs by heart work, but I choose to use calculator finally.
\item[-]Answer 2: Everywhere, all the time.

\item[4)]Do you have any personal preference for using a calculator?
\item[-]Answer 1: Yes, I like using a calculator with everything big for mutual interaction. I think that would be much more handy.
\item[-]Answer 2: I am OK with most designs for today's calculators.


\item[5)]If there is calculator, what kind of appearance would you like, say, the operators, keys, the button, the screen etc.?
\item[-]Answer 1: I hope it could be as big as my pocket, the buttons are no need to big but large screen. The operators no need to be very rich as I just need it to do some simple calculation like sum, substation, mod or multiplication. 
\item[-]Answer 2: not too big, easy to take, big screen with simple and clear buttons. For operators, four arithmetic operation is enough for me.


\item[6)]what do you think is the most necessary function with PI involved in this device?
\item[-]Answer 1: as I mentioned previously, circumference and area are often used, so I hope it has one-button circumference or area calculation function.
\item[-]Answer 2: to calculate circle area which is highly desired for me.


\item[7)]When you use the function, what other parameters do you think you are going to use?
\item[-]Answer 1: input radius or diameter, get circumference and area
\item[-]Answer 2: it is better to be able to record a group of parameter or numbers, so I can just find them in machine memory but no need to recall in head.

\item[8)]If the function is created, do you think what is the best way for you to use it conveniently?
\item[-]Answer 1: One-button calculations!
\item[-]Answer 2: Speech control perhaps...I don't want to type numbers all the time.

\end{itemize}

\section{Analysis}
\vspace{8pt}
\noindent
This interview gives an overview of  the expectation of two potential users for a calculator with a special number.  The interview was conducted with one domain professional and one common user.\\
 \hfill\break
Firstly, calculators are still broadly used, even for today with rapid development of technology, general-purpose calculators still have a very wide range of users.\\
 \hfill\break
Second, in addition to the desire for a more portable and easy-to-use design, it is also desirable to incorporate new technologies such as speech recognition.\\
 \hfill\break
Third, for special numbers, users are not satisfied with the function only to obtain the value itself, but hope to extract the calculation results related to other most used function employing a specific special number. For example, when the value of the PI is obtained, one button can also be used to obtain values of the circumference length and the circle area with PI.